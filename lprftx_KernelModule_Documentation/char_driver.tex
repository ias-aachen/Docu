% Char Driver 

\chapter{Char Driver}

Ein Char Driver zeichnet sich durch die Kommunikation zwischen dem Kernel und dem User Space aus. Es ist also m�glich Daten vom und zum User Space zu �bertragen. 

Im LPRF Treiber wird die Implementierung eines Char Drivers genutzt, um Registerwerte im laufenden Betrieb des Chips auszulesen. Dies erm�glicht ein leichteres Debuggen des Treibers. 
\urlstyle{same}


Wie bereits oben erw�hnt, wird der Char Driver mit Hilfe der Funktion \textit{\url{scull\_init\_module()}} in \textit{lprf\_probe()} initialisiert. 


\section{scull\_read()}

Um die Verbindung zwischen User Space und Kernel Module zu schaffen, spielt die Funktion \textit{scull\_read()} eine wichtige Rolle. Damit diese Funktion allerdings Daten an den User Space zur�ckgeben kann, muss dort eine Ger�tedatei mit der zum Treiber geh�rigen Major Number erstellt werden (\textit{sudo mknod /dev/lprf -c MajorNumber MinorNumber}). Im Anschluss ist es m�glich, diese Datei auszulesen und der Char Driver liefert an diese einen Wert zur�ck. Falls das Auslesen einfach �ber \textit{\$cat} geschieht, wird in dem hier vorliegendem Fall immer das erste Register ausgegeben. Um die Registeradresse frei zu w�hlen, kann das Programm \textit{userspace.c} genutzt werden. 


Wenn die Ger�tedatei vom User Space aus ausgelesen wird, wird in dem Char Driver die Funktion \textit{scull\_read()} ausgef�hrt (Listing \ref{lis:scullRead}). Der Parameter \textit{*buf} enth�lt die Informationen �ber den Ort, von dem die Funktion aufgerufen wurde. Somit kann der Treiber die Daten an den korrekten Ort im User Space zur�ck liefern. Mit der Variable \textit{*f\_pos} wird die Adresse des auszulesenden Registers festgelegt. Dieser Pointer kann im User Space mit der c-Funktion \textit{fseek} modifiziert werden. 


\begin{lstlisting}[language={C}, caption=Implementierung scull\_read(), label=lis:scullRead]
ssize_t scull_read(struct file *filp, char __user *buf, size_t count, loff_t *f_pos)
{
	ssize_t retval = 0;
	unsigned int myval = 1;
	unsigned int dev_size = 2 + *f_pos;
	unsigned int rc;
	struct lprf_state_change *ctx = &(lp->debug);

	printk(KERN_DEBUG "lprf: scull_read - start.");
	
	// read from hardware via spi_sync()
	lprf_sync_read_reg_debugging(lp, *f_pos, ctx, 0);

	// get read value
	myval = ctx->buf[2];	
	
	count = 1;			
	
	rc = copy_to_user(buf, &myval, count);	
	if (rc) {		//1 char is transfered, count = 1	
		retval = -EFAULT;
		goto out;
	}
	
	*f_pos += count;	//+= 1
		
	if (*f_pos + count > dev_size){		
		count = dev_size - *f_pos;
		printk(KERN_DEBUG "lprf: scull_read: count=%u\n", count);
		return 0;
	}

	retval = count;

out:
	printk(KERN_DEBUG "lprf: scull_read - end. %s:%i\n", __FILE__, __LINE__);	
	return retval;
	return 0;
}
\end{lstlisting}

F�r das Auslesen des Registers wird die Funktion \textit{lprf\_sync\_read\_debugging()} verwendet. Diese spezielle Funktion ist notwendig, damit bei der SPI Kommunikation der Code des LPRF Treibers unterbrochen wird und nicht weiter ausgef�hrt wird. Ansonsten w�rde \textit{scull\_write} fortgesetzt werden, wenn noch keine ausgelesenen Daten in ctx->buf[2] vorliegen w�rde. 

Nachdem das entsprechende Register ausgelesen wurde, ist es m�glich den Wert aus {ctx->buf[2]} in \textit{myval} zu speichern. Zum Schluss kann der Wert an den User Space zur�ckgegeben werden, wozu die Funktion \textit{copy\_to\_user()} genutzt wird. 
